\chapter{系统配置与控制}

\section{控制引脚说明}

龙芯 3A 的控制引脚总共包括 DO\_TEST、 ICCC\_EN、 NODE\_ID[1:0]、
CLKSEL[15:0]、PCI\_CONFIG: 它们的设置及位域含义见表~\ref{tab:sysPinControl}。

\begin{longtable}{|c|c|l|}
  \caption{控制引脚说明}\label{tab:sysPinControl} \\
  \hline 信号 & 上下拉 & 描述 \\ \hline\hline
  \endfirsthead

  \caption{控制引脚说明(续)} \\
  \hline 信号 & 上下拉 & 描述 \\ \hline\hline
  \endhead

  \multicolumn{3}{r}{\tiny 未完待续} \endfoot \endlastfoot

  DO\_TEST & 上拉 & 1: 功能模式; 0: 测试模式 \\ \hline

  \multirow{2}{*}{ICCC\_EN} & \multirow{2}{*}{下拉} & 1: 多芯片一致性互联模式(使用HT0互联); \\
                            &                       & 0: 单芯片模式 \\ \hline

  NODE\_ID[1:0] & & 在多芯片一致性互连模式下表示处理器号 \\ \hline

  CLKSEL[15:0] & & 
    \begin{tabular}{c}
      上电时钟控制 \\
      \begin{tabular}{|c|c|l|} \hline
        模块 & 信号  & \multicolumn{1}{c|}{作用} \\ \hline\hline
        HT   & \multirow{2}{*}{15} & 1: 采用内部参考电压 \\
             &       & 0: 采用外部参考电压 \\ \cline{2-3}
             & \multirow{2}{*}{14} & 1: HT PLL 采用差分时钟输入 \\
             &       & 0: HT PLL 采用普通时钟输入 \\ \cline{2-3}
             & \multirow{4}{*}{13:12} & 00: PHY 时钟为 1.6GHZ/1 \\
             &                        & 01: PHY 时钟为 3.2GHZ/2 \\
             &                        & 10: PHY 时钟为普通输入时钟 \\
             &                        & 11: PHY 时钟为差分输入时钟 \\ \cline{2-3}
             & \multirow{3}{*}{11:10} & 00: HT 控制器时钟 200MHz\\
             &                        & 01: HT 控制器时钟 400MHz\\
             &                        & 1x: HT 控制器时钟为普通输入时钟 \\ \hline
        MEM  & \multirow{3}{*}{9:5}   & 11111: MEM 时钟直接采用 memclk \\
             &                        & 其它: \\
             &                        & memclk*(clksel[8:5]+30)/(clksel[9]+3) \\ \hline
        CORE & \multirow{3}{*}{4:0}   & 11111: CORE 时钟直接采用 sysclk \\
             &                        & 其它: \\
             &                        & sysclk*(clksel[3:0]+30)/(clksel[4]+1) \\ \hline
        \cmcol{3}{} \\[-2.9ex] % hack to obtain a little space
      \end{tabular}
    \end{tabular} \\ \hline

  PCI\_CONFIG[7:0] & &
    \begin{tabular}{c}
      IO 配置控制 \\
      \begin{tabular}{|c|l|} \hline
        信号 & 说明 \\ \hhline
        PCI\_CONFIG[7] & HT 控制信号引脚电压控制位 1 \\
        PCI\_CONFIG[6:5] & PCIX 总线速度选择 \\
        PCI\_CONFIG[4] & PCIX 总线模式选择 \\
        PCI\_CONFIG[3] & PCI 主设备模式 \\
        PCI\_CONFIG[2] & 系统从 PCI 空间启动 \\
        PCI\_CONFIG[1] & 使用外部 PCI 仲裁 \\
        PCI\_CONFIG[0] & HT 控制信号引脚电压控制位 0 \\ \hline
      \end{tabular} \\
      具体设置 \\
      \begin{tabular}{cc}
        \begin{tabular}{|c|c|c|l|} \hline
          6 & 5 & 4 & PCI 模式速度 \\ \hhline
          0 & 0 & 0 & PCI 33/66 \\
          0 & 1 & 1 & PCI-X 66 \\
          1 & 0 & 1 & PCI-X 100 \\
          1 & 1 & 1 & PCI-X 133 \\ \hline
          \cmcol{3}{} \\[-2.9ex] % hack to obtain a little space
        \end{tabular} &
        \begin{tabular}{|c|c|c|} \hline
          7 & 0 & HT 信号引脚电压 \\ \hhline
          0 & 0 & 3.3v \\
          0 & 1 & 2.5v \\
          1 & 0 & 1.8v \\
          1 & 1 & 保留 \\ \hline
          \cmcol{3}{} \\[-2.9ex] % hack to obtain a little space
        \end{tabular}
      \end{tabular}
    \end{tabular} \\[.1cm] \hline
\end{longtable}

\section{芯片配置及采样寄存器}

龙芯 3A 中的芯片配置寄存器(Chip\_config)及芯片采样寄存器(chip\_sample)
提供了对芯片的配置进行读写的机制。\remark{What's the width of these registers?
BTW, more explanations are needed.}

\begin{longtable}{|c|c|c|c|p{6.5cm}|}
  \caption{芯片配置寄存器}\label{tab:chipConfigSample} \\
  \hline 位域 & 字段名 & 访问 & 复位值 &  描述 \\ \hhline \endfirsthead
  \caption{芯片配置寄存器(续)} \\
  \hline 位域 & 字段名 & 访问 & 复位值 &  描述 \\ \hhline \endhead

  \hline \multicolumn{5}{r}{\tiny 未完待续} \endfoot
  \endlastfoot

  \multicolumn{5}{|l|}{芯片配置寄存器(物理地址 0x1fe00180)} \\* \hline
    其它    & ---                      & R  &       & 保留                                                        \\ \hline
    34:32   & HT\_freq\_scale\_ctrl0   & RW & 111   & HT 控制器 1 分频                                            \\ \hline
    31:29   & HT\_freq\_scale\_ctrl0   & RW & 111   & HT 控制器 0 分频                                            \\ \hline
    28:24   & DDR\_Clksel              & RW & 11111 & 软件配置 DDR 时钟倍频关系
                                                      \footnote{当 DDR\_Clksel\_en 为 1 时有效。} \\ \hline
    19      & DDR\_reset1              & RW & 1     & 软件 reset DDR 控制器1                                      \\ \hline
    18      & DDR\_reset0              & RW & 1     & 软件 reset DDR 控制器0                                      \\ \hline
    17      & Mc1\_en                  & RW & 1     & 是否启用 DDR 控制器1                                        \\ \hline
    16      & Mc0\_en                  & RW & 1     & 是否启用 DDR 控制器0                                        \\ \hline
    15      & Core3\_en                & RW & 1     & 是否启用处理器核 3                                          \\ \hline
    14      & Core2\_en                & RW & 1     & 是否启用处理器核 2                                          \\ \hline
    13      & Core1\_en                & RW & 1     & 是否启用处理器核 1                                          \\ \hline
    12      & Core0\_en                & RW & 1     & 是否启用处理器核 0                                          \\ \hline
    9       & DDR\_buffer\_cpu         & RW & 0     & 是否打开 DDR 读访问缓冲                                     \\ \hline
    8       & Disable\_ddr2\_confspace & RW & 0     & 是否禁用 DDR 配置空间                                       \\ \hline
    3       & DDR\_Clksel\_en          & RW & 0     & 是否使用软件配置 DDR 倍频                                   \\ \hline
    2:0     & Freq\_scale\_ctrl        & RW & 111   & 处理器核分频                                                \\* \hhline
    \multicolumn{5}{|l|}{芯片采样寄存器(物理地址 0x1fe00190)} \\* \hline
    其它    &                          & R  &          & 保留                                                        \\ \hline
    111     & Thsens1\_overflow        & R  &          & 温度传感器 1 温度上溢(超过 128 度)                        \\ \hline
    110:104 & Thsens1\_out             & R  &          & 温度传感器 1 温度                                           \\ \hline
    103     & Thsens0\_overflow        & R  &          & 温度传感器 0 温度上溢(超过 128 度)                        \\ \hline
    102:96  & Thsens0\_out             & R  &          & 温度传感器 0 温度                                           \\ \hline
    57:56   & Bad\_ip\_ht              & R  &          & 2 个 HT 控制器是否坏                                        \\ \hline
    53:52   & Bad\_ip\_ddr             & R  &          & 2 个 DDR 控制器是否坏                                       \\ \hline
    51:48   & Bad\_ip\_core            & R  &          & 4 个处理器核是否坏                                          \\ \hline
    47:32   & Sys\_clksel              & R  &          & 板上倍频设置                                                \\ \hline
    31:16   & Pad3v3\_ctrl             & RW & 16'h780  & 3v3pad 控制                                                 \\ \hline
    15:0    & Pad2v5\_ctrl             & RW & 16'h780  & 2v5pad 控制                                                 \\ \hline
\end{longtable}

\section{系统物理地址空间分布}

龙芯 3 号系列处理器的系统物理地址分布采用全局可访问的层次化寻址设
计,以保证系统开发的扩展兼容。系统支持的物理地址宽度为 48 位,并按照地址的高 4
位被均匀分布到 16 个结点上,如表~\ref{tab:sysnodeaddr} 所示。每个结点分配了
44 位物理地址空间。

\begin{longtable}{|c|>{\tt}c|>{\tt}c|>{\tt}c|}
  \caption{系统全局节点级地址分布}\label{tab:sysnodeaddr} \\
  \hline 节点号 & 地址位[47:44] & 起始地址 & 结束地址 \\ \hhline \endfirsthead

  \caption{系统全局节点级地址分布(续)} \\
  \hline 节点号 & 地址位[47:44] & 起始地址 & 结束地址 \\ \hhline \endhead

  \hline \rmcol{4}{\tiny 未完待续} \endfoot
  \hline \endlastfoot

  0  & 0x0 & 0x0000\_0000\_0000 & 0x0FFF\_FFFF\_FFFF \\
  1  & 0x1 & 0x1000\_0000\_0000 & 0x1FFF\_FFFF\_FFFF \\
  2  & 0x2 & 0x2000\_0000\_0000 & 0x2FFF\_FFFF\_FFFF \\
  3  & 0x3 & 0x3000\_0000\_0000 & 0x3FFF\_FFFF\_FFFF \\
  4  & 0x4 & 0x4000\_0000\_0000 & 0x4FFF\_FFFF\_FFFF \\
  5  & 0x5 & 0x5000\_0000\_0000 & 0x5FFF\_FFFF\_FFFF \\
  6  & 0x6 & 0x6000\_0000\_0000 & 0x6FFF\_FFFF\_FFFF \\
  7  & 0x7 & 0x7000\_0000\_0000 & 0x7FFF\_FFFF\_FFFF \\
  8  & 0x8 & 0x8000\_0000\_0000 & 0x8FFF\_FFFF\_FFFF \\
  9  & 0x9 & 0x9000\_0000\_0000 & 0x9FFF\_FFFF\_FFFF \\
  10 & 0xA & 0xA000\_0000\_0000 & 0xAFFF\_FFFF\_FFFF \\
  11 & 0xB & 0xB000\_0000\_0000 & 0xBFFF\_FFFF\_FFFF \\
  12 & 0xC & 0xC000\_0000\_0000 & 0xCFFF\_FFFF\_FFFF \\
  13 & 0xD & 0xD000\_0000\_0000 & 0xDFFF\_FFFF\_FFFF \\
  14 & 0xE & 0xE000\_0000\_0000 & 0xEFFF\_FFFF\_FFFF \\
  15 & 0xF & 0xF000\_0000\_0000 & 0xFFFF\_FFFF\_FFFF \\
\end{longtable}

龙芯 3A 的单节点采用 4 核配置,每个处理器芯片上集成的 DDR
内存控制器、 HT 总线、PCI 总线等将分享该结点分到的 44 位物理地址空间。
在节点内部, 这些地址空间通过 44 位物理地址的高 3 位([43:41])
均匀分布到结点内连接的 8 个设备。 如表~\ref{tab:nodeAddr},
其中前四个块被分给了四个二级 Cache 模块, 而其余四块则分别分配给了连接在东南西北
4 个方向端口的设备。 
\begin{table}[h]
  \centering
  \begin{tabular}{|c|c|c|c|} \hline
    设备       & 地址[43:41]位  & 节点内起始地址    & 节点结束地址 \\ \hhline
    二级 Cache & 0,1,2,3        & 0x000\_0000\_0000 & 0x7FF\_FFFF\_FFFF \\ \hline
    东         & 4              & 0x800\_0000\_0000 & 0x9FF\_FFFF\_FFFF \\ \hline
    南         & 5              & 0xA00\_0000\_0000 & 0xBFF\_FFFF\_FFFF \\ \hline
    西         & 6              & 0xC00\_0000\_0000 & 0xDFF\_FFFF\_FFFF \\ \hline
    北         & 7              & 0xE00\_0000\_0000 & 0xFFF\_FFFF\_FFFF \\ \hline
  \end{tabular}
  \caption{节点内物理地址分布}
  \label{tab:nodeAddr}
\end{table}
\newline 举例说明如下:
\begin{verbatim}
    节点 0 的东端口设备的基地址为: 0x0800_0000_0000;
    节点 1 的南端口设备的基地址为: 0x1A00_0000_0000。
\end{verbatim}
其他节点,其他方向设备的地址空间依次类推。 根据芯片和系统结构配置的不同,
如果某端口上没有连接 从设备,则对应的地址空间保留,不允许访问。

龙芯 3A 的四个二级 Cache 模块分享了一个 43 位的地址空间。
为了平衡对每个 Cache 模块的访问以提高性能, 在龙芯 3A 上,可以根据实际应用的访问特性,
通过设置 SCID\_SEL 寄存器(地址:0x3FF00400)决定二级 Cache
的交叉寻址方式。
这种交叉寻址方式设定地址位的``特定两位''确定地址对应的二级 Cache 模块,
也就是说,一定大小的地址块被轮流交错地对应到四个二级 Cache 模块中去。
表~\ref{tab:scidsel} 列出了 SCID\_SEL 寄存器值与地址选择位的对应关系。
\begin{table}[htbp]
  \centering
  \begin{tabular}{|c|c||c|c|} \hline
    SCID\_SEL & 地址位选择 & SCID\_SEL & 地址位选择 \\ \hhline
    0x0       & 6:5        & 0x8       & 23:22      \\ \hline
    0x1       & 9:8        & 0x9       & 25:24      \\ \hline
    0x2       & 11:10      & 0xA       & 27:26      \\ \hline
    0x3       & 13:12      & 0xB       & 29:28      \\ \hline
    0x4       & 15:14      & 0xC       & 31:30      \\ \hline
    0x5       & 17:16      & 0xD       & 33:32      \\ \hline
    0x6       & 19:18      & 0xE       & 35:34      \\ \hline
    0x7       & 21:20      & 0xF       & 37:36      \\ \hline
  \end{tabular}
  \caption{节点内地址地址选择位设置}
  \label{tab:scidsel}
\end{table}

SCID\_SEL 寄存器可以通过软件动态修改。 缺省情况下,寄存器值为 0, 地址位 [6:5]
位被用来散列二级 Cache 的访问, 即 [6:5] 两位决定地址对应的二级 Cache
编号。\remark{there is a jump between 0x0 and 0x1?}

\section{地址路由配置} \label{sec:htAddrRoute}

龙芯 3A 的路由主要通过两级交叉开关(X1 和 X2)实现。 每一级交叉开关
有若干主端口,每个主端口都对应着 8 个地址窗口。
交叉开关负责对每个主端口接收到的请求进行路由配置,完成它们地址窗口的目标路由选择。

每个地址窗口的设置通过三个 64 位寄存器 BASE、MASK 和 MMAP 实现:其中,BASE
是地址基地址, 至少以 K 字节对齐;MASK 采用类似网络掩码,高位为 1 的格式;MMAP 是
映射地址基地址, 同时 MMAP 的低三位表示对应目标从端口的编号,MMAP[4]
表示是否允许取指,MMAP[5]表示是否允许块读,MMAP[7]表示窗口是否使能。
窗口命中及地址转换公式为:
\begin{verbatim}
      窗口命中: IN_ADDR & MASK == BASE
      输出地址: OUT_ADDR = (IN_ADDR & ~MASK) | {MMAP[63:10] || 10'h0}
\end{verbatim}

除了这 8 个地址窗口外, 龙芯 3A 还对每个主端口提供了一个``系统默认路由''。
这些地址窗口相互独立,并从配置窗口 0 到配置窗口
7,优先级依次下降。系统默认路由有最低的优先级。 按优先级顺序,
首先命中的窗口将对地址进行路由,所以默认配置路由只有在所有 8
个配置窗口都没有对命中的情况下才会生效。同时,这也就是说,在没有
对地址窗口进行配置前,所有的读写请求都会按照系统默认路由的设定进行。

而对于二 级交叉开关, 如果不允许 Cache 访问或取指访问的从端口可以将 MMAP[4] 或
MMAP[5] 设为 0。

\subsection{一级交叉开关地址路由}

一级交叉开关对应 8 个主端口, 而每个主端口都拥有 8 个地址窗口。
表~\ref{tab:X1MasterWinBases} 列出了一级交叉开关的 8
个主端口的窗口寄存器基地址,及其连接的设备。
\begin{table}[htbp]
  \centering
  \begin{tabular}{|c|c|c||c|c|c|} \hline
    主端口 & 端口设备 & 基地址       & 主端口 & 端口设备 & 基地址       \\ \hhline
    0      & Core0    & 0x3FF0\_2000 & 4      & ---      & 0x3FF0\_2400 \\ 
    1      & Core1    & 0x3FF0\_2100 & 5      & ---      & 0x3FF0\_2500 \\ 
    2      & Core2    & 0x3FF0\_2200 & 6      & HT0      & 0x3FF0\_2600 \\ 
    3      & Core3    & 0x3FF0\_2300 & 7      & HT1      & 0x3FF0\_2700 \\ \hline
  \end{tabular}
  \caption{一级交叉开关主端口窗口寄存器基地址}
  \label{tab:X1MasterWinBases}
\end{table}

\newpage
每个端口的 8 个地址窗口的配置寄存器相对基地址的偏移在表~\ref{tab:AXIWinOffset}
列出。 龙芯 3A 一级交叉开关缺省采用固定路由, 在上电启动时,
这些配置窗口都处于关闭状态,需要软件对其进行使能配置。
\begin{table}[htbp]
  \centering
  \begin{tabular}{|c|c||c|c||c|c|} \hline
    偏移 & 寄存器     & 偏移 & 寄存器     & 偏移 & 寄存器      \\ \hhline
    0x00 & WIN0\_BASE & 0x40 & WIN0\_MASK & 0x80 & WIN0\_MMAP  \\
    0x08 & WIN1\_BASE & 0x48 & WIN1\_MASK & 0x88 & WIN1\_MMAP  \\
    0x10 & WIN2\_BASE & 0x50 & WIN2\_MASK & 0x90 & WIN2\_MMAP  \\
    0x18 & WIN3\_BASE & 0x58 & WIN3\_MASK & 0x98 & WIN3\_MMAP  \\
    0x20 & WIN4\_BASE & 0x60 & WIN4\_MASK & 0xA0 & WIN4\_MMAP  \\
    0x28 & WIN5\_BASE & 0x68 & WIN5\_MASK & 0xA8 & WIN5\_MMAP  \\
    0x30 & WIN6\_BASE & 0x70 & WIN6\_MASK & 0xB0 & WIN6\_MMAP  \\
    0x38 & WIN7\_BASE & 0x78 & WIN7\_MASK & 0xB8 & WIN7\_MMAP  \\ \hline
  \end{tabular}
  \caption{交叉开关地址窗口寄存器地址偏移表}
  \label{tab:AXIWinOffset}
\end{table}

表~\ref{tab:X1Slave} 列出了一级交叉开关的从端口, 即接受请求的设备端。
同时,映射二级 Cache 的地址还受到 SCID\_SEL 寄存器的影响:只有地址窗口和
SCID\_SEL 的条件都满足的情况下,地址访问才可得以进行。
\begin{table}[htbp]
  \centering
  \begin{tabular}{*{2}{|c|>{\centering}p{2.5cm}|}} \hline
    从端口 & 端口设备     & 从端口 & 端口设备 \tabularnewline \hhline
    0      & 二级 Cache 0 & 4      & ---      \tabularnewline 
    1      & 二级 Cache 1 & 5      & ---      \tabularnewline 
    2      & 二级 Cache 2 & 6      & HT0      \tabularnewline 
    3      & 二级 Cache 3 & 7      & HT1      \tabularnewline \hline
  \end{tabular}
  \caption{一级交叉开关从端口设备}
  \label{tab:X1Slave}
\end{table}

表~\ref{tab:X1defaultAddrWin} 列出了一级交叉开关的系统默认路由配置。简单说来,
一级交叉开关的系统默认路由就是直接将主端口地址映射到对应的从端口,地址值不变。
系统默认路由映射到二级 Cache 的地址也同样受到 SCID\_SEL 寄存器的影响。
\begin{table}[htbp]
  \centering
  \begin{tabular}{|c|c|c|} \hline
  起始地址 & 结束地址 & 目标 \\ \hline
  0x0000\_0000\_0000 & 0x07FF\_FFFF\_FFFF & 二级Cache \\ \hline
  0x0C00\_0000\_0000 & 0x0DFF\_FFFF\_FFFF & HyperTransport 0 \\ \hline
  0x0E00\_0000\_0000 & 0x0FFF\_FFFF\_FFFF & HyperTransport 1 \\ \hline
  \end{tabular}
  \caption{一级交叉开关:系统默认路由配置}
  \label{tab:X1defaultAddrWin}
\end{table}

\newpage
\noindent 注意,在进行一级交叉开关的主端口窗口设置时,
\begin{itemize}
  \item MMAP[4] 与 MMAP[5] 必须为 1;
  \item 如果使用一级交叉开关对二级 Cache 地址进行映射,映射后的地址
    (即``从端口地址'')必须与映射前地址(即``主端口地址'')保持一致: 这是维护
    Cache 一致性的需要。 而映射到 HyperTransport 地址的配置不受这个约束限制。
\end{itemize}

\subsection{二级交叉开关地址路由}

龙芯 3A 的二级交叉开关是供 CPU 和 PCI 两个具有主功能的 IP 进
行路由选择和地址转换而设置的。 从 3A 的连接结构可以看出, 来自 CPU 的访问是从
二级 Cache 发出的。 这两个主设备都拥有 8 个地址窗口,完成目标地
址空间的选择以及从源地址空间到目标地址空间的转换。
表~\ref{tab:X2MasterWinBases} 列出了这两个主设备的窗口寄存器基地址。
\begin{table}[htbp]
  \centering
  \begin{tabular}{|c|c||c|c|} \hline
    主端口 & 基地址       & 主端口 & 基地址       \\ \hline
    CPU    & 0x3FF0\_0000 & PCI    & 0x3FF0\_0100 \\ \hline
  \end{tabular}
  \caption{二级交叉开关主端口窗口寄存器基地址}
  \label{tab:X2MasterWinBases}
\end{table}
\newline 对每个主设备而言,二级交叉开关有和一级开关相同的寄存器偏移,
如表~\ref{tab:AXIWinOffset} 所示。

二级交叉开关的目的包括 DDR 地址空间、 PCI
等慢速外设地址空间,和配置寄存器块(Xconf)共三个 IP 相关的地址空间。 3A
的实现中使用了两个内存控制器,这些模块的标号对应关系如表~\ref{tab:X2SlaveLabel}
所示。 二级交叉开关的系统默认路由是将所有的地址转送到系统配置寄存器模块,
即从端口 3。
\begin{table}[htbp]
  \centering
  \begin{tabular}{|c|c|} \hline
    从端口 & 端口设备                \\ \hhline
    0      & 0 号 DDR2/3 控制器      \\ 
    1      & 1 号 DDR2/3 控制器      \\ 
    2      & 低速 I/O(PCI,LPC 等) \\ 
    3      & 配置寄存器模块          \\ \hline
  \end{tabular}
  \caption{二级交叉开关:从端口设备编号}
  \label{tab:X2SlaveLabel}
\end{table}

与一级交叉开关相比,二级交叉开关的配置更加灵活。
一级交叉开关的窗口配置必须保证不对需要 Cache 一致性维护的(即二级 Cache)
访问进行地址转换,否则二级 Cache 的地址会与一级 Cache 的地址发生冲突,而导致
Cache 一致性维护错误。

表~\ref{tab:X2DefAddrConfig} 列出了系统启动时, 二级交叉开关的缺省地址窗口
配置值(其他寄存器缺省值皆为 0)。
\begin{table}
  \centering
  \begin{tabular}{|c|l|c|} \hline
    寄存器          & \cellalign{c|}{寄存器含义} & 启动缺省值                 \\ \hhline
    CPU\_WIN0\_BASE & CPU 窗口 0 基地址   & {\tt 0x0000\_0000\_0000\_0000} \\ 
    CPU\_WIN0\_MASK & CPU 窗口 0 掩码     & {\tt 0xFFFF\_FFFF\_F000\_0000} \\ 
    CPU\_WIN0\_MMAP & CPU 窗口 0 新基地址 & {\tt 0x0000\_0000\_0000\_00F0} \\ \hline
    CPU\_WIN1\_BASE & CPU 窗口 1 基地址   & {\tt 0x1000\_0000\_0000\_0000} \\ 
    CPU\_WIN1\_MASK & CPU 窗口 1 掩码     & {\tt 0xFFFF\_FFFF\_F000\_0000} \\ 
    CPU\_WIN1\_MMAP & CPU 窗口 1 新基地址 & {\tt 0x0000\_0000\_1000\_00F2} \\ \hline
    PCI\_WIN0\_BASE & PCI 窗口 0 基地址   & {\tt 0x0000\_0000\_8000\_0000} \\ 
    PCI\_WIN0\_MASK & PCI 窗口 0 掩码     & {\tt 0xFFFF\_FFFF\_8000\_0000} \\ 
    PCI\_WIN0\_MMAP & PCI 窗口 0 新基地址 & {\tt 0x0000\_0000\_0000\_00F0} \\ \hline
  \end{tabular}
  \caption{二级交叉开关:缺省地址窗口配置值}
  \label{tab:X2DefAddrConfig}
\end{table}

\newpage\noindent
根据缺省的寄存器配置,芯片启动后,
\begin{center}
  \begin{tabular}{|c|c|c|c|} \hline
    主端口 & 主端口地址              & 目标从端口   & 从端口地址         \\ \hline
    CPU    & \verb+0x0000_0000-0x0FFF_FFFF+ & 内存控制器 0 & \verb+0x0000_0000 - 0x0FFF_FFFF+ \\ \hline
    CPU    & \verb+0x1000_0000-0x1FFF_FFFF+ & PCI          & \verb+0x1000_0000 - 0x1FFF_FFFF+ \\ \hline
    PCI    & \verb+0x8000_0000-0x8FFF_FFFF+ & 内存控制器 0 & \verb+0x0000_0000 - 0x0FFF_FFFF+ \\ \hline
  \end{tabular}
\end{center}

此外,当出现由于 CPU 猜测执行引起对非法地址的读访问,所有地址窗口
都不命中时,由访存模块将返回全 0 的数据给 CPU,以防止 CPU 无限等待。

