\chapter{二级 Cache}

二级 Cache 模块是与 GS464 处理器 IP 配套设计的模块。该模块既可以和 GS464 对接,使 GS464 成为包括二级 Cache 在内的处理器 IP;也可以通过 AXI 网络连接多个 GS464 以及多个二级 Cache 模块, 形成片内多处理器的 CMP 结构。 二级 Cache 模块的主要特征包括:
\begin{itemize}
  \item 采用 128 位 AXI 接口;
  \item 8 项 Cache 访问队列;
  \item 关键字优先;
  \item 接收读失效请求到返回数据最快 8 拍;(\remark{最慢多少拍?})
  \item 通过目录支持 Cache 一致性协议;
  \item 可用于片上多核结构,也可直接和单处理器 IP 对接;
  \item 软 IP 级可配置二级 Cache 的大小(512KB/1MB) ;
  \item 采用四路组相联结构;
  \item 运行时可动态关闭;
  \item 支持 ECC 校验;
  \item 支持 DMA 一致性读写和预取读;
  \item 支持 16 种二级 Cache 散列方式;
  \item 支持按窗口锁二级 Cache;
  \item 保证读数据返回原子性。(\remark{写数据保证原子性吗?})
\end{itemize}

二级 Cache 模块包括二级 Cache 管理模块 (scachemanage) 及二级 Cache 访问 模块
(scacheaccess)。二级 Cache 管理模块负责处理器来自处理器和 DMA 的访问请 求,而二级
Cache 的 TAG、目录和数据等信息存放在二级 Cache 访问模块中。为降低功耗,二级
Cache 的 TAG、目录和数据可以分开访问,二级 Cache 状态位、w 位与 TAG
一起存储,TAG 存放在 TAG RAM 中,目录存放在 DIR RAM 中,数 据存放在 DATA RAM
中。失效请求访问二级 Cache,同时读出所有路的 TAG、 目录和数据,并根据 TAG
来选出数据和目录。替换请求、重填请求和写回请求 只操作一路的 TAG、目录和数据。

为提高一些特定计算任务的性能,二级 Cache 增加了锁机制。落在被锁区域 中的二级
Cache 块会被锁住, 因而不会被替换出二级 Cache (除非四路二级 Cache 
中都是被锁住的块)。通过 confbus 可以对二级 Cache
模块内部的四组锁窗口寄存器进行动态配置, 但必须保证四路二级 Cache
中一定有一路被锁住。每组窗口 的大小可以根据 mask 进行调整,但不能超过整个二级
Cache 大小的 3/4。此外 当二级 Cache 收到 DMA 写请求时, 如果被写的区域在二级
Cache 中命中且被锁 住,那么 DMA 写将直接写入到二级 Cache 而不是内存。

\begin{table}
  \centering
  \begin{tabular}{|l|c|l|l|} \hline
    名称          & 地址       & 位域    & 描述 \\ \hhline
    slock0\_valid & 0x3FF00200 & [63:63] & 0 号锁窗口有效位 \\
    slock0\_addr  & 0x3FF00200 & [47:0]  & 0 号锁窗口锁地址 \\
    slock0\_mask  & 0x3FF00240 & [47:0]  & 0 号锁窗口掩码 \\ \hline
    slock1\_valid & 0x3FF00208 & [63:63] & 1 号锁窗口有效位 \\
    slock1\_addr  & 0x3FF00208 & [47:0]  & 1 号锁窗口锁地址 \\
    slock1\_mask  & 0x3FF00248 & [47:0]  & 1 号锁窗口掩码 \\ \hline
    slock2\_valid & 0x3FF00210 & [63:63] & 2 号锁窗口有效位 \\
    slock2\_addr  & 0x3FF00210 & [47:0]  & 2 号锁窗口锁地址 \\
    slock2\_mask  & 0x3FF00250 & [47:0]  & 2 号锁窗口掩码 \\ \hline
    slock3\_valid & 0x3FF00218 & [63:63] & 3 号锁窗口有效位 \\
    slock3\_addr  & 0x3FF00218 & [47:0]  & 3 号锁窗口锁地址 \\
    slock3\_mask  & 0x3FF00258 & [47:0]  & 3 号锁窗口掩码    \\ \hline
  \end{tabular}
  \caption{二级 Cache 锁窗口寄存器配置}
  \label{tab:l2cachewinconfig}
\end{table}

举例来说,一个地址 addr,在slock0\_valid 使能的情形下,如果满足如下条件
\begin{verbatim}
  (addr & slock0_mask) == (slock0_addr & slock0_mask)
\end{verbatim}
那么这个地址就被锁窗口 0 锁住了。

