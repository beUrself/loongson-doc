\chapter{中断的配置及使用}

material largely overlaps with that of chapter 7. Question: move below to
there?

\section{中断的流程}

龙芯 3A 处理中断的流程,从外部中断请求到内核对中断的处理, 其过程都是一样的。
下图给出的就是 3A-690e 板卡的中断处理的流程图。

图 11-1 3A-690e 中断流程图 (FIXME)

\section{中断路由及中断使能}

龙芯 3A 芯片最多支持 32 个中断源,以统一方式进行管理, 如
图~ref{fig:IntDispatch} (FIXME: use the one on chapter 7) 所示, 任意一个 IO
中断源可以被配置为是否使能、 触发的方式、以及被路由的目标处理器核中断脚。

\subsection{中断路由}

在龙芯 3A 中集成了 4 个处理器核,上述的 32 位中断源可以通过软件配置
选择期望中断的目标处理器核。进一步,中断源可以选择路由到处理器核中断 INT0 到
INT3 中的任意一个, 即对应 CP0\_Status 的 IP2 到 IP5。 也就是说上图所 示
CORE0~CORE3 的 IP0~IP3 对应的就是 CP0\_Status 的 IP2~IP5。 个 I/O 中断 32
源中每一个都对应一个 8 位的路由控制器,表~ref(tab in ch7): just describ there
is fine\ldots 路由寄存器采用向量的方式进行路由选择,如 0x48 标示路由到 3
号处理器的 INT2 上。
中断路由寄存器的说明

为了便于理解,下面会给出 3A-690e 和 3A-KD60 两个板卡在中断路由上的配置情况:

a) 3A-690e 硬件连接为“CPU 串口 + HT1 接南北桥”。路由设置为:
\lstinputlisting[language=C]{codes/3a690e-ht1-route.c}

b) 3A-KD60 硬件连接为“CPU 串口 + CPU 的 PCI + CPU”的 8259,路由设置为:
\lstinputlisting[language=C]{codes/3akd60-cpupci-route.c}

\subsection{中断使能}

中断相关配置寄存器都是以位的形式对相应的中断线进行控制, 中断控制位
连接及属性配置见表 11-3 (FIXME: quote the table in chapter
2)。中断使能(Enable)的配置有三个寄存器:Intenset、 Intenclr 和
Inten。Intenset 设置中断使能,Intenset 寄存器写 1 的位对应的中断被
使能。Intenclr 清除中断使能,Intenclr 寄存器写 1 的位对应的中断被清除。Inten
寄存器读取当前各中断使能的情况。脉冲形式的中断信号(如 PCI\_SERR)由 Intedge
配置寄存器来选择,写 1 表示脉冲触发,写 0 表示电平触发。中断处理 程序可以通过
Intenclr 的相应位来清除脉冲记录。

不仅需要使能 IO 的控制器, 具体到所接的 IO, 如果它有自己的中断控制器,
也需要单独使能,如 LPC 中断控制器,HT 中断控制器,具体的寄存器配置可查
看寄存器手册。下面列出了一些可能需要使能的中断控制器:

\begin{lstlisting}
  /* Enable the IO interrupt controller ,LPC(10) and HT(16~31)*/
  t = *(volatile unsigned int*)0x900000003ff01428;

  *(volatile unsigned int*)0x900000003ff01428 = t | (0xffff << 16) | (0x1 << 10);

  /* Enable LPC interrupt controller*/
  *(volatile unsigned int*)(0xffffffffbfe00200 + 0x00) = 0x80000000;

  /* the 18-bit interrpt enable bit */
  *(volatile unsigned int*)(0xffffffffbfe00200 + 0x04) = 0x0;

  /* Enable HT interrupt,only used 7 interrupt vectors*/
  *(volatile unsigned int*)0x90000EFDFB0000A0 = 0xffffff7f;
\end{lstlisting}

\section{中断分发}

中断发生时,硬件会设置 cause 寄存器的 Excode 域及相关的 IP 位。进而进
入软件处理过程,软件通过查询 Excode 域来确定是哪一种类型的异常,来选择
使用何种异常处理例程。如果是我们要讨论的硬件中断,就会进入平台相关的中
断分发函数。中断分发函数再根据 cause 寄存器的 IP 位及 IM 位(中断屏蔽)来
进行一级中断分发,然后再根据中断号的位域来进行二次分发,然后执行具体的 do\_IRQ
中断操作的处理。

内核在启动阶段,会将每个异常向量与每个异常处理例程对应起来。异常共有
32 种,在这里我们只讨论 0 号异常也就是硬件中断。在
trap\_init()函数中,异常的通用入口地址设置为 0x80000180,
该地址保存了一个函数指针为 expect\_vec3\_generic,见内核
arch/mips/kernel/genex.S。expect\_vec3\_generic()函数 会根据取得的 Excode
码,进入相应的例程函数。在 trap\_init()中异常代码 0 即硬件中断与 handle\_int
例程相关,handle\_int 见内核 arch/mips/kernel/genex.S。hanle\_int
最终跳转到平台相关的 plat\_irq\_diapatch()中断分发函数,
进行中断的一级级分发。

以 3A-690e 为例,Cause 寄存器的 IP0 和 IP1
对应的是软中断,IP6 对应的是核间中断,IP7 对应的是时钟中断,IP2 对应的是 Cpu
的串口及 LPC 的中断,IP3 对应的是路由到 HT1 的中断。HT1 接北桥
690E,北桥再接南桥 SB600,南北桥的中断都由南桥上的 8259 控制, 各个外设如
USB、等的中断路由到 8259 sata 控制器见 arch/mips/kernel/fixup\_ev3a.c 中的
godson3a\_smbus\_fixup 函数,其中寄 存器的定义见 AMD 南桥手册之《AMD SB600
Register Reference Manual.pdf》。 pcibios\_map\_irq 函数是对 PCI 及 PCIE 槽
位的扫描及中断号分配,文件 fixup\_ev3a.c 中其他函数主要是对一些固定 PCI
设备的中断号进行分配,详细的中断分配及到 8259 路由的情况可以详见文件
fixup\_ev3a.c 的代码及注释,寄存器使用查看 AMD 的南桥手册。

中断分发完后,运行 do\_IRQ()函数,跳转到对应的具体驱动程序执行。

