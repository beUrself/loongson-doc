\chapter{地址窗口配置转换}

龙芯 3A 采用两级交叉开关结构,两级交叉开关窗口可分别配置,用于控制
将特定地址发往特定接收端进行处理。另外,HyperTransport 控制器内部也对芯
片对内及对外可访问的地址窗口有所控制。

\section{一二级交叉开关地址窗口配置方法}

交叉开关上每个主端口各拥有 8 个地址窗口可供配置。每个地址窗口由 BASE、MASK 和
MMAP 三个 64 位寄存器组成,BASE 以 K 字节对齐,即每 个地址窗口的所分空间最少为
1KB;MASK 为窗口掩码;MMAP 为窗口映射后
的地址,同时[2:0]表示对应目标从端口的编号,MMAP[4] 表示允许取指,MMAP [5]
表示允许块读,MMAP [7] 表示窗口使能。 窗口命中的判断如下:
\begin{displaymath}
  主端口地址 \& MASK == BASE
\end{displaymath}
映射后的从端口地址转换公式如下:
\begin{displaymath}
  从端口地址 = 主端口地址 \& (~(MASK)) | MMAP \& MASK
\end{displaymath}

\section{一级交叉开关地址窗口}

一级交叉开关拥有默认路由设置, 这个设置不被地址窗口配置寄存器所显示,
只有不在任意一个地址窗口命中的地址会被这个默认路由所解释。

每个主端口的地址窗口配置相互独立,各拥有 8 个可配置窗口。配置窗口优
先级依次下降,从配置窗口 0 开始,第一个命中的窗口对这个地址进行路由。优
先级次序如下表~\ref{tab:addrWinPriority}:
\begin{table}
  \centering
  \begin{tabular}{|c|c|} \hline
    优先级 &   窗口       \\ \hline
      最高 & 配置窗口0    \\
           & 配置窗口1    \\
           & 配置窗口2    \\
           & 配置窗口3    \\
           & 配置窗口4    \\
           & 配置窗口5    \\
           & 配置窗口6    \\
           & 配置窗口7    \\
      最低 & 系统默认窗口 \\ \hline
  \end{tabular}
  \caption{配置窗口路由优先级}
  \label{tab:addrWinPriority}
\end{table}

\section{一级交叉开关地址窗口配置时机}

一级交叉开关地址窗口配置, 对于需要路由至二级 Cache 请求的窗口配置要
求是非常严格的,在配置前后需要保证二级 Cache 与一级 Cache 的数据一致性,
也就是说,绝不允许在配置后出现如下情况:
\begin{emph}
  配置前二级 Cache 中拥有一个备份,一级 Cache 中也拥有相应的备份,但
  在配置后有关这个备份地址的请求将被路由到其它的从端口上。
\end{emph}
上述的这种情况最终将导致一二级 Cache 数据的错乱。因此,在配置各个窗
口的最佳时机是在系统还没有进行 Cache 操作之前。 其它的情况下如果需要对一
级交叉开关进行配置都必须保证不会出现上述情况。 需要注意的是,在进行 Cache
操作之后需改 scid\_sel 的值本身也会带来这种 不一致,因为地址空间与所映射的二级
Cache 号已经出现了改变。

\section{二级交叉开关地址窗口}

二级交叉开关同样拥有默认路由, 所有不被任一地址窗口所命中的地址都会

被路由至从端口 3,也即系统配置寄存器空间上。 对于二级交叉开关的主端口,
也即对外发出请求的主设备端, 包括如下几个:
\begin{itemize}
  \item 0 号主端口:二级 Cache 0-3
  \item 1 号主端口:PCI 主端口
\end{itemize}
对于二级交叉开关的从端口, 也即对外发出请求的从设备端, 包括如下几个:
\begin{itemize}
  \item 0 号从端口:内存控制器 0
  \item 1 号从端口:内存控制器 1
  \item 2 号从端口:低速设备接口,包括 PCI 从端口、 LPC、 UART、 SPI 接口、PCI
    寄存器空间
  \item 3 号从端口:系统配置寄存器
\end{itemize}

相对于一级交叉开关, 二级交叉开关的地址窗口配置的限制条件会比较弱一
些,主要由软件来保证重新配置地址窗口后的访问内容不会出现错误即可。 比如,
之前将系统地址的 0x0000\_0000\_0000 – 0x0000\_0FFF\_FFFF 映射到内 存控制器 0
上的 0x0000\_0000\_0000 – 0x0000\_0FFF\_FFFF,并使用这个地址进行
了一些读写操作,存储了一些有效数据。在对地址窗口进行配置之后,将系统地 址 的
0x0000\_2000\_0000 – 0x0000\_2FFFF\_FFFF 映 射 到 内 存 控 制 器 0 的
0x0000\_0000\_0000 – 0x0000\_0FFF\_FFFF。 此时对 0x0000\_2000\_0000 的访问会得
到原来使用 0x0000\_0000\_0000 地址存入的值。

在这个过程中, 需要注意的是二级 Cache 中的内容并没有根据地址窗口配置
的改变而改变,也就是说,如果原来对 0x0000\_0000\_0000 的写访问采用 Cache 方式,
那么很可能在地址窗口更新后对 0x0000\_20000\_0000 的访问会得到一个旧 值。

\section{对地址窗口配置的特别处理}

由于龙芯 3A 处理器核会向外发生一些猜测访问,这些猜测访问可能会落在
任意的地址空间。但是,并不是任何设备都允许被猜测访问,尤其是对于 PCI
设备来说,一个猜测的读访问很可能会造成一个“读清”寄存器数据的破坏,也
有可能造成一些对非法地址的访问无法正常返回,从而发生处理器死机的情况。
我们通过对一二级交叉开关的配置来防止这些情况的发生, 将一些不可猜测
访问的地址空间禁止掉。 例如我们通过对二级交叉开关进行下面的设置来防止对 PCI
空间 0x1000\_0000 的猜测访问,但同时允许对 0x1FC0\_0000 的猜测访问。

\begin{tabular}[h]{|c|c|c|c|} \hline
         & BASE                  & MASK                  & MMAP \\ \hhline
  窗口 0 & 0x0000\_0000\_1000\_0000 & 0xFFFF\_FFFF\_F000\_0000 & 0x0000\_0000\_1000\_0082 \\ \hline
  窗口 1 & 0x0000\_0000\_1FC0\_0000 & 0xFFFF\_FFFF\_FFF0\_0000 & 0x0000\_0000\_1FC0\_00F2 \\ \hline
\end{tabular}

对于一级交叉开关来说,对二级 Cache 地址的映射同时也受到 scid\_sel 的影
响,这两者不可以冲突。 如果需要将一个地址空间映射给二级 Cache 时, 需要考虑二级
Cache 散列的 影响。 即将该地址空间映射到各个二级 Cache
上。因为每个地址窗口只可以对应一个从端口,那么对二级 Cache
的映射就需要通过四个地址窗口映射来完成。 另外,因为地址窗口的最小单位是
1KB,所以如果此时对二级 Cache 空间 进行配置就需要将 scid\_sel 的值设为 2 以上,
即使用 10 位以上的地址进行散列。 如下表的例子,
对一级交叉开关进行配置来将处理器核发出的所有地址访问映射 至二级 Cache。
\begin{center}
  \begin{tabular}{|c|c|c|c|} \hline
    \multicolumn{4}{|l|}{scid\_sel = 2} \\ \hline
    & BASE                  & MASK                  & MMAP \\ \hhline
    窗口 4 & 0x0000\_0000\_0000\_0000 & 0x0000\_0000\_0000\_0C00 & 0x0000\_0000\_0000\_00F0 \\ \hline
    窗口 5 & 0x0000\_0000\_0000\_0400 & 0x0000\_0000\_0000\_0C00 & 0x0000\_0000\_0000\_04F1 \\ \hline
    窗口 6 & 0x0000\_0000\_0000\_0800 & 0x0000\_0000\_0000\_0C00 & 0x0000\_0000\_0000\_08F2 \\ \hline
    窗口 7 & 0x0000\_0000\_0000\_0C00 & 0x0000\_0000\_0000\_0C00 & 0x0000\_0000\_0000\_0CF3 \\ \hline
  \end{tabular}
\end{center}
由此窗口可知, 凡是没有在窗口 0-3 中命中的地址都将会在这 4 个窗口中命
中,并根据 scid\_sel 将整个地址空间散列至四个正确的二级 Cache 中。

\section{HyperTransport 地址窗口}

HyperTransport 控制器不仅可以向外发送读写访问,也可以实现外部设备对
处理器内部内存的 DMA 访问。这两个访问的访问地址空间相互独立,下面分别 介绍。

\subsection{处理器核对外访问地址窗口}

龙芯 3A 芯片共有 4 个 HyperTransport 控制器,分别为 HT0\_LO、HT0\_HI、
HT1\_LO、HT1\_HI,其中,HT0\_LO 与 HT0\_HI 共用一个物理接口,HT1\_LO 与 HT1\_HI
共用一个物理接口,当芯片引脚 HTx\_8x2 置为低时,只有 HTx\_LO 对 用户可见,而
HTx\_HI 的地址空间无效。按照一级交叉开关的默认路由,各个控 制器的地址空间如下:
\begin{center}
  \begin{tabular}{|c|c|c|c|c|}\hline
    基地址           &    结束地址      & 大小     &   定义      &   说明 \\ \hhline
    0x0C00\_0000\_0000 & 0x0CFF\_FFFF\_FFFF & 1 Tbytes & HT0\_LO 窗口 & \\ \hline
    0x0D00\_0000\_0000 & 0x0DFF\_FFFF\_FFFF & 1 Tbytes & HT0\_HI 窗口 & HT0\_8x2 =1 时有效 \\ \hline
    0x0E00\_0000\_0000 & 0x0EFF\_FFFF\_FFFF & 1 Tbytes & HT1\_LO 窗口 & \\ hline
    0x0F00\_0000\_0000 & 0x0FFF\_FFFF\_FFFF & 1 Tbytes & HT1\_HI 窗口 & HT1\_8x2 =1 时有效 \\ \hline
  \end{tabular}
\end{center}

每个 HyperTransport 控制器内拥有 40 位地址空间,按照 HyperTransport 协议,
表~\ref{tab:htaddrspace} 列出了这 40 位地址的划分。 其中 MEM 空间、I/O 空间、HT
总线配置空间分别对应于传统 PCI 空间上 的三种访问形式,分别是 PCI MEM 访问、PCI
IO 访问与 PCI 配置访问。HT 控 制器配置空间主要提供 HT
中断向量及内部窗口配置等功能。 HT
总线配置空间,按外部设备的“总线号”“设备号”“功能号”“寄存 、 、 、
器偏移”按图~\ref{fig:htconfig}表示的规则对设备配置地址进行直接读写访问。

\subsection{外部设备对处理器芯片内存DMA访问地址窗口}

为了保护处理器芯片内的内存数据,HyperTransport 控制器为外部设备的 DMA
访问提供了一组窗口,即用户手册中 9.5.4 节的“接收地址窗口” ,只有落
在这组窗口中的 DMA 地址才会被真正对芯片内的内存空间进行操作, 否则会给
发起该访问的外设做出错误应答。 这组窗口由下面的几个参数决定,窗口命中规则如下:
\begin{verbatim}
 hit = ht_rx_image_en &&
         (bus_addr & ht_rx_image_mask ) == (ht_rx_image_base & ht_rx_image_mask)
 addr_out = ht_rx_image_trans_en ?
         ht_rx_image_trans | bus_addr & ~ht_rx_image_mask : bus_addr
\end{verbatim}
不同的地址窗口优先级依次下降。

\section{地址空间配置实例分析}

下面针对 PMON 里面对两级交叉开关的配置分别予以说明。 以下的实例中, HT 设备使用
HT1 接口连接,HT0 接口悬空不用。

\subsection{一级交叉开关实例 1}

\noindent 一级交叉开关的一种配置如下:
\begin{center}
  \begin{tabular}[h]{|c|c|c|c|} \hline
    & BASE                     & MASK                     & MMAP \\ \hhline
    窗口 0 & 0x0000\_0000\_1800\_0000 & 0xFFFF\_FFFF\_FC00\_0000 & 0x0000\_0EFD\_FC00\_00F7 \\ \hline
    窗口 1 & 0x0000\_0000\_1000\_0000 & 0xFFFF\_FFFF\_F800\_0000 & 0x0000\_0E00\_1000\_00F7 \\ \hline
    窗口 2 & 0x0000\_0000\_1E00\_0000 & 0xFFFF\_FFFF\_FF00\_0000 & 0x0000\_0E00\_0000\_00F7 \\ \hline
    窗口 3 &          ---             &        ---               & \\ \hline
    窗口 4 & 0x0000\_0C00\_0000\_0000 & 0xFFFF\_FC00\_0000\_0000 & 0x0000\_0C00\_0000\_00F7 \\ \hline
    窗口 5 &          ---             &        ---               & \\ \hline
    窗口 6 & 0x0000\_1000\_0000\_0000 & 0x0000\_1000\_0000\_0000 & 0x0000\_1000\_0000\_00F7 \\ \hline
    窗口 7 & 0x0000\_2000\_0000\_0000 & 0x0000\_2000\_0000\_0000 & 0x0000\_2000\_0000\_00F7 \\ \hline
  \end{tabular}
\end{center}
下面一一分析每个配置窗口的作用。
\begin{enumerate}
  \item 窗口 0,将 0x1800\_0000 的地址转换为 0x0000\_0EFD\_FC00\_0000,并路由
    至 HT1 控制器。这样实际上将原来需要使用 64 位地址才能访问的 HT IO 空间、 HT
    配置空间直接使用 32 位地址空间进行映射,使用映射之后的这些空间使用 32
    位地址即可访问。
    \begin{center}
      \begin{tabular}{|c|c|c|l|} \hline
               & 转换前地址               & 转换后地址               & 说明 \\ \hline
        地址 0 & 0x0000\_0000\_18xx\_xxxx & 0x0000\_0EFD\_FCxx\_xxxx & HT IO 空间 \\ \hline
        地址 1 & 0x0000\_0000\_19xx\_xxxx & 0x0000\_0EFD\_FDxx\_xxxx & HT IO 空间 \\ \hline
        地址 2 & 0x0000\_0000\_1Axx\_xxxx & 0x0000\_0EFD\_FExx\_xxxx & HT 配置空间:Type 0 \\ \hline
        地址 3 & 0x0000\_0000\_1Bxx\_xxxx & 0x0000\_0EFD\_FFxx\_xxxx & HT 配置空间:Type 1 \\ \hline
      \end{tabular}
    \end{center}

  \item 窗口 1,将 0x1000\_0000 的地址转换为 0x0000\_0E00\_1000\_0000,并路由至
    HT1 控制器。 这样实际上将原来需要使用 64 位地址才能访问的 HT MEM 空间的
    一部分直接使用 32 位地址空间进行映射, 使用映射之后的这些空间使用 32 位地
    址即可访问。虽然没有映射全部的 HT MEM 空间,但在这里最多已经可以使用 多达
    128MB 的 HT MEM 空间了。
    \begin{center}
      \begin{tabular}{|c|c|c|l|} \hline
               & 转换前地址               & 转换后地址               & 说明 \\ \hline
        地址 0 & 0x0000\_0000\_1xxx\_xxxx & 0x0000\_0E00\_1xxx\_xxxx & HT MEM 空间 \\ \hline
      \end{tabular}
    \end{center}

  \item 窗口 2,将 0x1E00\_0000 的地址转换为 0x0000\_0E00\_0000\_0000,并路由至
    HT1 控制器。 这样实际上将原来需要使用 64 位地址才能访问的 HT MEM 空间的
    最低 16MB 地址直接使用 32 位地址空间进行映射,使用映射之后的这些空间使 用
    32 位地址即可访问。之所以要映射这部分地址,是因为一些传统设备需要使
    用这部保留空间进行固定的译码,例如显卡设备等等。
    \begin{center}
      \begin{tabular}{|c|c|c|l|} \hline
               & 转换前地址               & 转换后地址               & 说明 \\ \hline
        地址 0 & 0x0000\_0000\_1Exx\_xxxx & 0x0000\_0E00\_1Exx\_xxxx & HT MEM 空间低 16MB \\ \hline
      \end{tabular}
    \end{center}

    前面这几个窗口的设置是为了在 PMON 里面可以直接使用 32 位地址, 不经 TLB
    转换即可实现 HT 空间及 HT 设备的访问,以方便软件版本的兼容设计,在 linux
    系统里面,因为可以直接使用 64 位地址进行访问,所以并不需要这些地址
    转换。但是需要注意的是,基于 linux 对 HT 的 MEM 空间的处理方式,所以依
    然需要 HT MEM 空间的转换来简化对 32 位地址寻址的外设的访问。

  \item 剩余的窗口用于将没有响应设备的地址全部路由至 HT1 控制器,由 HT1 控
    制器进行响应。 这些地址在正常的程序执行过程中并不会主动出现,但由于处理
    器核的猜测执行, 任何地址的访问都有可能出现,如果得不到正确响应则有可能
    造成处理器死机。龙芯 3A 中的 HT 控制器可以正确识别并处理此类访问。因此
    需要将这些潜在的猜测访问全部路由至 HT 控制器。

  \item 除了这些地址之外的其它地址都会根据默认路由方法路由至二级
    Cache,再向二级交叉开关转发。
\end{enumerate}

\subsection{一级交叉开关实例 2}

\noindent 一级交叉开关的另一种配置如下 (scid\_sel=1):
\begin{center}
  \begin{tabular}[h]{|c|c|c|c|} \hline
    & BASE                     & MASK                     & MMAP \\ \hhline
    窗口 0 & 0x0000\_0000\_1800\_0000 & 0xFFFF\_FFFF\_FC00\_0000 & 0x0000\_0EFD\_FC00\_00F7 \\ \hline
    窗口 1 & 0x0000\_0000\_1000\_0000 & 0xFFFF\_FFFF\_F800\_0000 & 0x0000\_0E00\_1000\_00F7 \\ \hline
    窗口 2 & 0x0000\_0000\_1E00\_0000 & 0xFFFF\_FFFF\_FF00\_0000 & 0x0000\_0E00\_0000\_00F7 \\ \hline
    窗口 3 & 0x0000\_0E00\_0000\_0000 & 0xFFFF\_FE00\_0000\_0000 & 0x0000\_0E00\_0000\_00F7 \\ \hline
    窗口 4 & 0x0000\_0000\_0000\_0000 & 0x0000\_0000\_0000\_0C00 & 0x0000\_0000\_0000\_00F0 \\ \hline
    窗口 5 & 0x0000\_0000\_0000\_0400 & 0x0000\_0000\_0000\_0C00 & 0x0000\_0000\_0000\_04F1 \\ \hline
    窗口 6 & 0x0000\_0000\_0000\_0800 & 0x0000\_0000\_0000\_0C00 & 0x0000\_0000\_0000\_08F2 \\ \hline
    窗口 7 & 0x0000\_0000\_0000\_0C00 & 0x0000\_0000\_0000\_0C00 & 0x0000\_0000\_0000\_0CF3 \\ \hline
  \end{tabular}
\end{center}
这种配置情况下,前 3 个窗口与前一种配置相同,这里不再赘述。 窗口 3 将
0x0000\_0E00\_0000\_0000 的地址全部路由至 HT1
控制器上,这本是默认的一种路由方式,在这里进行这个配置是因为在窗口 4-7
中,将所有的地址都路由至四个不同的二级 Cache 中,所以默认的路由将不再生效。
窗口 4-7 的配置方式也在 1.5
节中已经解释过一遍,其中最重要的地方在于\emph{路由至各个二级 Cache 的方式应该与
scid\_sel 的配置一致}。这种配置的目的同第一种配置相同,
都是为了防止一些没有设备响应的地址导致处理器死机。

\subsection{二级交叉开关实例 1}

\noindent 二级交叉开关的一种配置如下。这种配置下只使用一个内存控制器。使用内
存上的 256MB 空间。
\begin{center}
  \begin{tabular}{|c|c|c|c|} \hline
           &   BASE                &   MASK                &   MMAP \\ \hline
    窗口 0 & 0x0000\_0000\_1000\_0000 & 0xFFFF\_FFFF\_F000\_0000 & 0x0000\_0000\_1000\_0082 \\ \hline
    窗口 1 & 0x0000\_0000\_1FC0\_0000 & 0xFFFF\_FFFF\_FFF0\_0000 & 0x0000\_0000\_1FC0\_00F2 \\ \hline
    窗口 2 & 0x0000\_0000\_0000\_0000 & 0xFFFF\_FFFF\_F000\_0000 & 0x0000\_0000\_0000\_00F0 \\ \hline
  \end{tabular}
\end{center}
其他的窗口(3-7)都没有打开。
\begin{enumerate}
  \item 窗口 0 打开了低速设备空间的 uncache 且非取指的访问,这样就可以保证落
    在这个窗口的访问都是程序需要作出的访问。
  \item 窗口 1 打开了低速设备空间中 BOOT 空间的所有类型访问,即包括 cache
    访问和取指访问在内的正常访问,猜测访问可以对这个空间进行正常访问。
  \item 窗口 2 打开了内存控制器 0 上的低 256MB 空间,允许所有类型的访问。
  \item 除此之外的所有访问都将按照默认路由被路由至系统配置寄存器空间。
\end{enumerate}

与 1.8.1 (FIXME) 中的一级交叉开关配置相结合,得到的全芯片地址空间如下:
\begin{center}
  \begin{tabular}{|c|c|c|l|} \hline
            & 起始地址                 &     结束地址             & 说明 \\ \hhline
    地址 0  & 0x0000\_0000\_0000\_0000 & 0x0000\_0000\_0FFF\_FFFF & 内存控制器 0 \\ \hline
    地址 1  & 0x0000\_0000\_1000\_0000 & 0x0000\_0000\_17FF\_FFFF & HT1 MEM 空间 \\ \hline
    地址 2  & 0x0000\_0000\_1800\_0000 & 0x0000\_0000\_19FF\_FFFF & HT1 IO 空间 \\ \hline
    地址 3  & 0x0000\_0000\_1A00\_0000 & 0x0000\_0000\_1BFF\_FFFF & HT1 配置空间 \\ \hline
    地址 4  & 0x0000\_0000\_1C00\_0000 & 0x0000\_0000\_1DFF\_FFFF & LPC Memory \\ \hline
    地址 5  & 0x0000\_0000\_1FC0\_0000 & 0x0000\_0000\_1FCF\_FFFF & LPC Boot \\ \hline
    地址 6  & 0x0000\_0000\_1FD0\_0000 & 0x0000\_0000\_1FDF\_FFFF & PCI IO 空间 \\ \hline
    地址 7  & 0x0000\_0000\_1FE0\_0000 & 0x0000\_0000\_1FE0\_00FF & PCI 控制器配置空间 \\ \hline
    地址 8  & 0x0000\_0000\_1FE0\_0100 & 0x0000\_0000\_1FE0\_01DF & IO 寄存器空间 \\ \hline
    地址 9  & 0x0000\_0000\_1FE0\_01E0 & 0x0000\_0000\_1FE0\_01E7 & UART 0 \\ \hline
    地址 10 & 0x0000\_0000\_1FE0\_01E8 & 0x0000\_0000\_1FE0\_01EF & UART 1 \\ \hline
    地址 11 & 0x0000\_0000\_1FE0\_01F0 & 0x0000\_0000\_1FE0\_01FF & SPI \\ \hline
    地址 12 & 0x0000\_0000\_1FE0\_0200 & 0x0000\_0000\_1FE0\_02FF & LPC Register \\ \hline
    地址 13 & 0x0000\_0000\_1FE8\_0000 & 0x0000\_0000\_1FE8\_FFFF & PCI 配置空间 \\ \hline
    地址 14 & 0x0000\_0000\_1FF0\_0000 & 0x0000\_0000\_1FF0\_FFFF & LPC I/O \\ \hline
    地址 15 & 0x0000\_0C00\_0000\_0000 & 0x0000\_0FFF\_FFFF\_FFFF & HT1 控制器,各种空间 \\ \hline
    地址 16 & 0x0000\_1000\_0000\_0000 & 0x0000\_3FFF\_FFFF\_FFFF & HT1 控制器,猜测空间 \\ \hline
    地址 17 & 其它地址                 &                          & 系统配置空间 \\ \hline
  \end{tabular}
\end{center}

\subsection{二级交叉开关实例 2}

\noindent 二级交叉开关的另一种配置如下。这种配置下使用两个内存控制器。每个内
存控制器使用 1GB 的内存空间
\begin{center}
  \begin{tabular}{|c|c|c|c|} \hline
           & BASE                     & MASK                     & MMAP \\ \hhline
    窗口 0 & 0x0000\_0000\_1000\_0000 & 0xFFFF\_FFFF\_F000\_0000 & 0x0000\_0000\_1000\_0082 \\ \hline
    窗口 1 & 0x0000\_0000\_1FC0\_0000 & 0xFFFF\_FFFF\_FFF0\_0000 & 0x0000\_0000\_1FC0\_00F2 \\ \hline
    窗口 2 & 0x0000\_0000\_0000\_0000 & 0xFFFF\_FFFF\_F000\_0000 & 0x0000\_0000\_0000\_00F0 \\ \hline
    窗口 3 &                          &                          & \\ \hline
    窗口 4 & 0x0000\_0000\_8000\_0000 & 0xFFFF\_FFFF\_C000\_0000 & 0x0000\_0000\_0000\_00F0 \\ \hline
    窗口 5 &                          &                          & \\ \hline
    窗口 6 & 0x0000\_0000\_C000\_0000 & 0xFFFF\_FFFF\_C000\_0000 & 0x0000\_0000\_0000\_00F1 \\ \hline
    窗口 7 &                          &                          & \\ \hline
  \end{tabular}
\end{center}
\begin{enumerate}
  \item 窗口 0 打开了低速设备空间的 uncache 且非取指的访问,这样就可以保证落
    在这个窗口的访问都是程序需要作出的访问。
  \item 窗口 1 打开了低速设备空间中 BOOT 空间的所有类型访问,即包括 cache
    访问和取指访问在内的正常访问,猜测访问可以对这个空间进行正常访问。
  \item 窗口 2 打开了内存控制器 0 上的低 256MB 空间,允许所有类型的访问。
  \item 窗口 4 打开了内存控制器 0 上的所有 1GB 空间,系统使用 0x8000\_0000 –
    0xBFFF\_FFFF 的地址进行访问,需要注意的是,系统 0x8000\_0000 –
    0x8FFF\_FFFF 的空间与 0x0000\_0000 – 0x0FFF\_FFFF
    的空间相重合,为了保证数据的正确性,系统软件必须保证仅使用其中一种地址对这部为进行访问,以
    linux 为例,必须使用系统中 0x0000\_0000 – 0x0FFFF\_FFFF
    的地址,那么,对于这个空间,0x8000\_0000 - 8FFF\_FFFF
    的访问就是被禁止的。
  \item 窗口 6 打开了内存控制器 1 上的所有 1GB 空间,系统使用 0xC0000\_0000 –
    0xFFFF\_FFFF 对这段内存进行访问。
\end{enumerate}

与 1.8.2 (FIXME) 中的一级交叉开关配置相结合,得到的全芯片地址空间分布如下:


\begin{center}
  \begin{tabular}{|c|c|c|l|} \hline
            & 起始地址                 & 结束地址                 & 说明 \\ \hhline
    地址 0  & 0x0000\_0000\_0000\_0000 & 0x0000\_0000\_0FFF\_FFFF & 内存控制器 0 \\ \hline
    地址 1  & 0x0000\_0000\_1000\_0000 & 0x0000\_0000\_17FF\_FFFF & HT1 MEM 空间 \\ \hline
    地址 2  & 0x0000\_0000\_1800\_0000 & 0x0000\_0000\_19FF\_FFFF & HT1 IO 空间 \\ \hline
    地址 3  & 0x0000\_0000\_1A00\_0000 & 0x0000\_0000\_1BFF\_FFFF & HT1 配置空间 \\ \hline
    地址 4  & 0x0000\_0000\_1C00\_0000 & 0x0000\_0000\_1DFF\_FFFF & LPC Memory \\ \hline
    地址 5  & 0x0000\_0000\_1FC0\_0000 & 0x0000\_0000\_1FCF\_FFFF & LPC Boot \\ \hline
    地址 6  & 0x0000\_0000\_1FD0\_0000 & 0x0000\_0000\_1FDF\_FFFF & PCI IO 空间 \\ \hline
    地址 7  & 0x0000\_0000\_1FE0\_0000 & 0x0000\_0000\_1FE0\_00FF & PCI 控制器配置空间 \\ \hline
    地址 8  & 0x0000\_0000\_1FE0\_0100 & 0x0000\_0000\_1FE0\_01DF & IO 寄存器空间 \\ \hline
    地址 9  & 0x0000\_0000\_1FE0\_01E0 & 0x0000\_0000\_1FE0\_01E7 & UART 0 \\ \hline
    地址 10 & 0x0000\_0000\_1FE0\_01E8 & 0x0000\_0000\_1FE0\_01EF & UART 1 \\ \hline
    地址 11 & 0x0000\_0000\_1FE0\_01F0 & 0x0000\_0000\_1FE0\_01FF & SPI \\ \hline
    地址 12 & 0x0000\_0000\_1FE0\_0200 & 0x0000\_0000\_1FE0\_02FF & LPC Register \\ \hline
    地址 13 & 0x0000\_0000\_1FE8\_0000 & 0x0000\_0000\_1FE8\_FFFF & PCI 配置空间 \\ \hline
    地址 14 & 0x0000\_0000\_1FF0\_0000 & 0x0000\_0000\_1FF0\_FFFF & LPC I/O \\ \hline
    地址 15 & 0x0000\_0000\_8000\_0000 & 0x0000\_0000\_BFFF\_FFFF & 内存控制器 0 \\ \hline
    地址 16 & 0x0000\_0000\_C000\_0000 & 0x0000\_0000\_FFFF\_FFFF & 内存控制器 1 \\ \hline
    地址 17 & 0x0000\_0E00\_0000\_0000 & 0x0000\_0FFF\_FFFF\_FFFF & HT1 控制器,各种空间 \\ \hline
    地址 18 & 其它地址                 &                          & 系统配置空间 \\ \hline
  \end{tabular}
\end{center}

