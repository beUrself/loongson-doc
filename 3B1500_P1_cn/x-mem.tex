\chapter{X系统的内存分配}

X系统的内存分配问题,描述的就是显卡的显存分配问题。在3A-690e中,显卡是集成在北桥
690E内部,是PCIE的一个设备。该显卡的显示核心是ATI X1250,内部集成了128M的显存,
也支持共享显存的方式,共享显存最大也可达128M。显卡的显示过程是这样的:CPU将有关
作图的指令和数据通过PCIE总线传送给显卡。GPU再根据CPU的要求,完成图像处理过程,并
将最终图像数据保存在显存中。
 
图16 1显卡处理图像显示的过程

对于使用独立显存的情况来说,由于显存是在显卡内部,过程会变得比较简单,显卡可以直
接将内容写入显存。而对于共享显存来说,过程会较为复杂一点,GPU把要写入的显存地址
告诉CPU之后,会有两种情况:一、显存地址是个PCI空间地址。CPU会把内容再直接写到
PCIE总线上,PCIE再做一次地址转换,转换到实际的显存地址上,也就是物理地址上;二、
显存地址就是内存地址。这时CPU会把要内容直接写入内存中的显存位置。显然,对于共享
内存的方式来说,采用第二种方案,效率上会更高。

下面具体讲下共享显存的第二种方式在PMON中式如何实现的。为了扩展我们的PCI空间,我
们使用的TLB映射。在 PMON 的 bonito.h 中,我们是这样定义的:

\lstinputlisting[language=C]{codes/bonito.pciiobase.h}

意思是把256M的PCI空间(0x10000000~0x20000000)分成的两个部分:
0x10000000~0x17ffffff为mem空间,0x18000000~0x20000000为IO空间。而mem空间的虚拟地
址0xd0000000到物理地址0x10000000是通过手动填充TLB来实现转换的。在3A-690e中,如果
内存为2G,显存的PCI地址其实就是0x10000000,对应的内存地址是0xf8000000,如果内存为
1G,显存的PCI地址其实就是0x10000000,对应的内存地址是0x78000000,这个也是通过TLB
映射来实现的。TLB映射的代码如下:

\lstinputlisting[language={[MIPS]Assembler}]{codes/tlb.map.S}

这样,在显卡访问显存时,我们都可以使用0xf8000000的地址来作为显存的起始地址,这样
它实际上使用的就是内存的高端部分了。这个是在rs690\_struct.c中ati\_nb\_cfg的结构
体中设置的,把system\_memory\_tom\_lo设成0x1000,也就是0x100000000,如果显存是128M
,那么起始地址就是0x1000M-128M=0xf8000000的地址,而这个地址就是上面所说的显存的
虚拟地址,根据TLB映射可以得到实际的显存物理地址。至此,显卡直接访问显存就是这么
实现的。

