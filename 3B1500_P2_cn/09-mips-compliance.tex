\chapter{MIPS 兼容性}

龙芯 GS464 处理器的设计目标是要兼容 MIPS64 R2 架构,这个目标已经基本实现。
在设计过程,出于对性能、将来扩展的考虑以及具体实现的原因,龙芯 GS464 核与 MIPS64 R2 相比
在细节上有一些不太重要的差异上。 同时, GS464 在 MIPS64 R2 的基础上实现了若干扩展。以下,
我们将从指令集构架和特权资源构架两个方面对这些差异进行阐述。

\section{指令集架构}

在指令集架构(Instruction Set Architecture,简称 ISA)上,
龙芯 GS464 处理器的实现了 MIPS64 R2 的所有指令, 不过在一些指令的实现上
做了重定义,具体细节见\ref{sec:gsSpecialInstruction} 节。
同时, 龙芯 GS464 处理器在 MIPS64 R2 的基础上实现了若干扩展。
这些增强指令包括如下五个方面:
\begin{itemize}
  \item 扩展定点乘除法指令, 扩大了 MIPS64 的 定点乘除法指令,
    以提高定点乘法运算使用广泛的应用性能。 这些指令主要 执行 64
    位整点乘除法操作,并产生 64 位(而不是 128 位)结果。 附录 A
    对龙芯 GS464 的扩展整数指令进行了详细介绍。
  \item 扩展浮点指令,这些指令采用了三个操作数模式,而不是MIPS64
    中的四操作数形式。 附录 B龙芯 GS464 扩展浮点指令进行了详细介绍。
  \item 扩展多媒体指令,即单指令多数据(SIMD)指令,
    对高性能媒体和通信应用提供了强大支持。这些指令类似于 x86 平台的 SSE
    的指令。 附录~C 对龙芯 GS464 的多媒体指令进行了详细介绍。
  \item 扩展访存指令 -- to be added.
  \item x86 虚拟指令 -- to be added.
\end{itemize}

\subsection{CPU 特殊实现指令列表}\label{sec:gsSpecialInstruction}

龙芯 GS464 处理器核对所有 MIPS64 R2 指令都作了支持。但对一些与实现
相关的指令作了重定义,见表

\begin{longtable}{|@{\hspace{.2cm}}c@{\hspace{.2cm}}|l|l|}
  \caption{龙芯 CPU 特殊实现指令}\label{tab:special-impl-ins} \\
  \hline 指令 OpCode & \cellalign{c|}{描述} & \cellalign{c|}{龙芯具体实现} \\ \hhline \endfirsthead
  \caption{龙芯 CPU 特殊实现指令(续)} \\
  \hline 指令 OpCode & \cellalign{c|}{描述} & \cellalign{c|}{龙芯具体实现} \\ \hhline \endhead
  \hline \multicolumn{3}{r}{\tiny 未完待续} \endfoot \endlastfoot

  PREF   & 预取指令     & 空操作,预取可通过 Load 到 0 号寄存器实现 \\
  PREFX  & 预取指令     & 空操作,预取可通过 Load 到 0 号寄存器实现 \\
  SSNOP  & 单发射空操作 & 空操作,GS464 实现了所有相关指令的硬件互锁 \\
  EHB    & 隔离执行相关 & 空操作,GS464 实现了所有相关指令的硬件互锁 \\
  WAIT   & 进入等待状态 & 空操作,GS464 实现了通过动态调频,让 CPU 等待或休眠 \\
  RDPGPR & 读影子寄存器 & GS464 没有影子寄存器,读当前寄存器 \\
  WRPGPR & 写影子寄存器 & GS464 没有影子寄存器,写当前寄存器 \\ \hline
\end{longtable}

\subsection{浮点转换指令}

龙芯 GS464 浮点运算中进行数据格式转换时, 比如使用如下指令将浮点数转换为
字整点数或双字整点数,
\begin{verbatim}
    cvt.w.fmt, round.w.fmt, floor.w.fmt, ceil.w.fmt, trunk.w.fmt
    cvt.l.fmt, round.l.fmt, floor.l.fmt, ceil.l.fmt, trunk.l.fmt
\end{verbatim}
如果输入值为负非数,负越界,负无穷时, 而 FCSR 寄存器的
无效例外没有使能, 将不会发出无效操作例外,
返回值为负最大:0x8000\_0000 或 0x8000\_0000\_0000\_0000 取决于操作的转换格式。
而在 MIPS64 浮点运算中,遇到相同情况时, 也不会发出无效操作例外,
不过返回值规定为正最大: 0x7fffffff或0x7fffffffffffffff。 即, MIPS64 的规定不
区分输入值的正负。

\section{特权资源构架}

龙芯 GS464 处理器实现了 MIPS64 特权资源架构的绝大部分, 而在一些地方则采用了
R10000 的特权资源架构,同时也引入了 一些特殊的实现特征 --- 这些特征和 MIPS64
的特权资源架构是不兼容的。 为了实现的灵活性, MIPS 在实践中也
允许一个兼容设计只实现部分的特权资源架构。 作为 MIPS64
定义的特权资源架构的重要部分, 这些差异主要体现在 CP0 上。
系统程序员在移植或编写软件时应充分考虑到这些不兼容的特征。
以下各节将更详细地描述这些主要的特征差异。

\subsection{Diagnostic 寄存器}

Diagnotic 寄存器是一个 64 位的辅助寄存器。该寄存器具有刷新 ITLB 、
BTB (分支目标缓冲器)和启用 RAS(返回地址堆栈)的功能。当在 Diagistic 寄存器
的 ITLB 位写入 1 时,整个 ITLB 将被刷新。

\subsection{ITLB 刷新}\label{subsec:itlb-flushing}

龙芯 GS464 包含了独立的 ITLB(即指令 TLB),以减少对
JTLB 的竞争及降低功耗。 当翻译指令地址发生 ITLB 查找脱靶时,
将从 JTLB 中查找命中项,并随机替换一个 ITLB 项。这里的 ITLB
的查找和替换操作对用户是完全透明的。
然而, GS464 处理核没有提供自动刷新 ITLB 的功能:当 JTLB 表项映射的地址
被 TLBWI 或 TLBWR 指示改变时,ITLB 不能自动更新以保持与 JTLB 的
一致。 因此,系统程序员必须使用指令刷新 ITLB。 基于同样的原因,
当任何 JTLB 项的位域被改变时,比如失效一个页面或页面的掩码大小
发生变化时,也必须用软件刷新 ITLB。

刷新 ITLB 是通过设置 CP0 的 Diagistic 寄存器的 ITLB 位实现的。
ITLB 项不能单独地被刷新: 这个操作将丢弃 ITLB 中的所有项。
以下给出了一个刷新 ITLB 的例子。

\lstinputlisting{codes/itlb-flushing.S}

\subsection{异常返回(Status[ERL] = 1)}

ERET 指令用于从异常,中断或错误陷阱中返回。当复位,软
复位,NMI或缓存错误异常发生时,处理器将设置 ERL 位。对于
MIPS64 兼容处理器,在 Status[ERL]=1 的情况下, ERET 指令将清零
ERL位 并从 ErrorEPC 寄存器的值返回。龙芯 GS464 处理器的处理略有不同:
ERL 位被设置时,ERET 指令将会清除该位, 并依照 EPC
(而不是 ErrorEPC)寄存器给出的地址返回。

\subsection{64 位地址空间}

MIPS64 兼容处理器一般同时支持 32 位和 64 位地址空间:不同
的操作模式可以通过 CP0 状态寄存器的 UX、SX 和 KX
位设置。 而龙芯 GS464 处理器的 64 位地址空间总是启用的。也就是说,
状态寄存器的 KX、SX或 UX 位的值始终为 1,软件不能将它们写为 0。
所以无论处理器运行在何种模式下,用户空间(XUSEG、XSUSEG 和 XKUSEG 段),
管理空间(XSSEG 和 XKSSEG 段)及内核空间(KXSEG 和
KXPHYS 段)始终有效。内存管理一章的 \ref{sec:addrspace} 节
对这些虚拟地址段给出了详细的描述。

由于龙芯 GS464 处理器只能工作在 64 位地址空间下,所以 TLB 重填例外只能使用 XTLB
例外向量。 XTLB 重填例外 在龙芯 GS464 处理器上有和 TLB 重填向量相同的位置,
即偏移量为 0x00。这是龙芯 GS464 处理器与其他 MIPS64 兼容处理器的重要差别。

\begin{tabular}{|c|p{6.5cm}|p{6cm}|} \hline
  项目 & GS464 & MIPS64 \\ \hhline
  TLB 例外 &
  龙芯 GS464 不支持KX,UX,SX位,XTLB例外入口与TLB例外入口相同 &
  MIPS64 中用 KX,UX,SX 位来区分 64 位地址和 32 位地址, 且 TLB 例外和 XTLB
  例外分别使用不同的例外入口。 \\ \hline 
\end{tabular}

